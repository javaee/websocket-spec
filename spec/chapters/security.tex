\chapter{Server Security}
\label{security}

Websocket endpoints are secured using the web container security model. The goal of this is to make it easy for a websocket developer to declare whether access to a websocket server endpoint needs to be authenticated, and who can access it, and if it needs an encrypted connection or not. A websocket which is mapped to a given \textbf{ws://} URI (as described in Chapters \ref{configuration} and \ref{annotations}) is protected in the deployment descriptor with a listing to a \textbf{http://} URI with same hostname, port and path since this is the URL of its opening handshake.
Accordingly, websocket developers may assign an authentication scheme, user roles granted access and transport guarantee to their websocket endpoints. 

\section{Authentication of Websockets}

This specification does not define a mechanism by which websockets themselves can be authenticated. Rather, by building on the servlet defined security mechanism, a websocket that requires authentication must rely on the opening handshake request that seeks to initiate a connection to be previously authenticated. Typically, this will be performed by a Http authentication (perhaps basic or form-based) in the web application containing the websocket prior to the opening handshake to the websocket.

If a client sends an unauthenticated opening handshake request for a websocket that is protected by the security mechanism, the websocket implementation must return a \textbf{401 (Unauthorized)} response to the opening handshake request and may not initiate a websocket connection [WSC-8.1-1].

\section{Authorization of Websockets}

A websocket’s authorization may be set by adding a \textbf{$<$security-constraint$>$} element to the \textbf{web.xml} of the web application in which it is packaged. The $<$url-pattern$>$ used in the security constraint must be used by the container to match the request URI of the opening handshake of the websocket [WSC-8.2-1]. The implementation must interpret any http-method other than GET (or the default, missing) as not applying to the websocket. [WSC-8.2-2]

\section{Transport Guarantee}

A transport guarantee of \textbf{NONE} must be interpreted by the container as allowing unencrypted \textbf{ws://} connections to the websocket [WSC-8.3-1]. A transport guarantee of \textbf{CONFIDENTIAL} must be interpreted by the implementation as only allowing access to the websocket over an encrypted (\textbf{wss://}) connection [WSC-8.3-2] This may require a pre-authenticated request. 

\section{Example}

This example snippet from a larger web.xml deployment descriptor shows a security constraint for a websocket endpoint. In the example, the websocket endpoint which matches on an incoming request URI of \textbf{`quotes/live'} relative to the context root of the containing web application is protected such that it may only be accessed using \textbf{wss://}, and is available only to authenticated users who belong either to the \textbf{GOLD\_MEMBER} or \textbf{PLATINUM\_MEMBER} roles.

\begin{listing}{1}
<security-constraint>
     <web-resource-collection>
          <web-resource-name>
			LiveQuoteWebSocket
		</web-resource-name>
          <description>
             Security constraint for
             live quote websocket endpoint
          </description>
          <url-pattern>/quotes/live</url-pattern>
          <http-method>GET</http-method>
     </web-resource-collection>
     <auth-constraint>
          <description>
            definition of which roles 
		  may access the quote endpoint
          </description>
          <role-name>GOLD_MEMBER</role-name>
          <role-name>PLATINUM_MEMBER</role-name>
     </auth-constraint>
     <user-data-constraint>
          <description>WSS required</description>
          <transport-guarantee>
		  CONFIDENTIAL
   		</transport-guarantee>
     </user-data-constraint>
</security-constraint>
\end{listing}
